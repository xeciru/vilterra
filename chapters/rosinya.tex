\documentclass[../main.tex]{subfiles} 

\begin{document}

%%%%%%%%%%%%%%%%%%%%%%%%%%%%%%%%%%%%%%%%%%%%%%%%%%%%%%%%%%%%%%%%%%%%%%%
%                               Intro                                 %
%%%%%%%%%%%%%%%%%%%%%%%%%%%%%%%%%%%%%%%%%%%%%%%%%%%%%%%%%%%%%%%%%%%%%%%

\emph{Rosinya}. A old kingdom, now growing into a grand empire. A land that is rapidly tranforming, 
at the cusp of a great change that will some consume the world. 
Where many may see opportunity, others only find the schackles of an oppressive regime.

What ever may occur in this most holy of domain, the whole world watches. 


%%%%%%%%%%%%%%%%%%%%%%%%%%%%%%%%%%%%%%%%%%%%%%%%%%%%%%%%%%%%%%%%%%%%%%%
%                               Geography                             %
%%%%%%%%%%%%%%%%%%%%%%%%%%%%%%%%%%%%%%%%%%%%%%%%%%%%%%%%%%%%%%%%%%%%%%%

\section{Geography} 
Sitting right on the center of vilterra's east coast. Rosinya enjoys a warm meditarean climate
with a rich land and coastal ecosystem. The country spreads across a diverse geographic
region which includes grasslands, temperate forests, low mountains, and ria coastlines.

\subsection{Weather} 
It is mostly sunny thoughout the years, exceptable for rain in the winter. 
This hosiptal environment is oft attributed to the grace of the revered 
storm god, \emph{Palid}. Although their might be an underlying current of his wrath,
as southern Rosinya has some of the roughest coastlines on the continent.
This tough enviroment has bred a powerful Naval culture which contributes 
to Rosinya's ability to project power overseas. 


%%%%%%%%%%%%%%%%%%%%%%%%%%%%%%%%%%%%%%%%%%%%%%%%%%%%%%%%%%%%%%%%%%%%%%%
%                               Ecology                               %
%%%%%%%%%%%%%%%%%%%%%%%%%%%%%%%%%%%%%%%%%%%%%%%%%%%%%%%%%%%%%%%%%%%%%%%

\section{Ecology}
%INSERT SOMETHING

\subsection{Fauna and Flora}

\subsubsection{the farmlands}
Due to rosinya's unmatch fertile soil, agricultural field prolifate the entire 
kingdom. These fields are dominanted by grains such as wheat, barely, oats and vegetables
such as cabbage, kale, and lettuce. However, Rosinya is also a major producer, 
consumer, and exporter of rice. 
(in fact exporting rice to the dwarves, has become a compitition between Rosinya and the Asakhan).

Agricultural land isn't just reserved for crops, as Ranching is also domiant in Rosinya.
Cows, chickens, and pigs are by far the most common farm animals. Cheese, milk, and Egg
based products are a long time stable in Rosinyean dishes. On special occassions pigs or
cows are slaughtered to provide the indegredents for steak and sausages.

In the past, farmlands fell victim to monster targeting livestock. However both independent 
monster hunter groups and an, increasingly interventitist, central government, have managered 
to stave these attacks.

\subsubsection{fishing and sea life}
Due to the abdunce of editable sea life, Rosinya has grown a large fishing indrustry.
Upon the stalls in major cities of Rosinya it isn't uncommon to find a diverse 
supply of seafood, including fish, squid, clams, and even octopus. Although not all
of these fish are avalible to everyone. Only the rich merchants, guild members, and nobles
enjoy such cuisine as octopus. Meanwhile, the average worker often eats markerel, cod, 
or sardines. This aquatic food culture even mades it deep into the countryside. Farmers would 
rathers eat the tuna caught downstream, than slaughter a cow that could otherwise 
provide valuable milk and cheese.

\subsection{Hostile Creatures}
Despite the best efforts of the Rosinyean government and independent monster hunters.
Many mysterious creatures still roam the deep woods of Rosinya.   

\subsubsection{subnatural phenomania}
For some inexplicable reason, in the west reaches of Rosinya undead and fiendish 
creatures occasional return to the mortal plane during the night. Their prowel around, 
some searching from victims to drag prematurely into the afterlife, others more 
innocenly, observe the joys of mortal life their now lack. 

\subsubsection{horror below and above the sea}
Well at the very least the kingdom can fight off fiends and necromany on land, the 
seas along Rosinya's coasts are a whole different story. Beneath the waves the 
demonic merrow, krakens, sea turtles, and the sort lurk, waiting to strike on unsuspecting 
victims sailing on the surface. There are many a tale of lonely fishermen being
dragged down by sirens, or travelling families being petrafied by swarming cockatrice.

On the rare occasion a hydra or sea turtle graces the shores, drawing the full
attention of Rosinya's mighty navy. 

With the advent of the caravel and blackpowder cannons, naval patrols have become more 
successful at challenging these threats. Yet, there is alway's air of caution 
when heading out to sea.

% \section{demographics} 

\section{internal politics}
Rosinya is a land expirencing rapid and unparalleled change. This is most evident within
it's mutanting govermental system and tumultuous domenstic politics.

\subsection{expanding monarchy}
The position of \emph{Rei Dom}, the lord king, has grown increasing more powerful. 
The current ruler \emph{João Braceo IV} and his precessors have taken steps to
centeralize the government and increase their own political power. More then ever, 
the monarch intevenes in local affairs, and has stunted the authority of the noblity.
The most drastic of these actions happen 18 years ago, when the former king \emph{Pedro Braceo II}
desolved the \emph{Cortes}.

This power shift has been aided by the establishment of the \emph{Inquistorius} and the 
backing of major Guilds. This rise of power has severally split the country. Many praise the 
strenghting of Rosinya due to the Kings divine guidance, while others have thrown 
accusations of tyranny.

% \section{geopolitics}


%%%%%%%%%%%%%%%%%%%%%%%%%%%%%%%%%%%%%%%%%%%%%%%%%%%%%%%%%%%%%%%%%%%%%%%
%                               history                               %
%%%%%%%%%%%%%%%%%%%%%%%%%%%%%%%%%%%%%%%%%%%%%%%%%%%%%%%%%%%%%%%%%%%%%%%

\section{history} 
Even the old Kingdom of Rosinya, is relatively young on the scale of Vilterra's anchient 
history. Yet the land itself has lived though it all.

\subsection{anchient history}
Like the rest of the continent, the lands now known by Rosinya were colonized by the 
Serovean Empire during the conquests of An Sa'ora. No records exist before this and even
records thoughout the rest of the Serovean's millennia reign are sparse. It is known that 
Wood Elven and Gnome tribes have always inhabitated forested parts of the region. High Elves
intially stuck to grandiose coastal cities, but would venture inland to magically terraform.
In fact the abducance of flowers in the region is likely due to Serovean engineering. 

The Serovean city of \emph{Tel Shoku} directly south of modern Nirviré, was by far the most 
populus settlement. It would serve much the same purpose of modern Nirviré; being a trade choke 
point between the north and south of Vilterra. 

\subsubsection{first human settlement}
The first humans to arrive in modern Rosinya were Nordic. They arrived a millinium and a half ago 
to escape famine caused by a mini ice age. At the time the Serovean's had collapsed into puesdo-anarchy
so human settlement remained relatively undisturbed. These groups would only form small clans, preforming 
raids against both Wood Elves and High Elves for resources.

It wouldn't be for two centuries until humans from the Archipelligo of Mitos arrived.
The Mitosian settlers were much more organized, bringing fleets of ships with the materials 
to form permeant cities. They also brought professial soldiers and were more knowledgable in the ways of 
magic. Detrathesis was the first of this settlements founded knew the modern city of Ajetos. Not long afterwards
the legendary naval general, Sirio, would found the settlement of Nirvidium, direct precussor of modern Nirviré.

To secure their new territory, Mitosian city states, allied themselves with the Serovean Empire. The Serovean High
Elves saw them as more Civilized then other human groups. They grant the Mitosians land, in exchange for doing 
the dirty work of repelling Nordic and Wood Elven incusions. Nirvidium and Detrathesis abused this relationship
to control the region, and insure future human domination.  

\subsubsection{The Meneketes Empire}
The year 8268 saw a monumental shift for the entirety of Vilterra. The Half-human Half-dragon king, Aeos Meneketes,
embarked on a continent spaning conquest, and Rosinya was his first target. In April 8268, he employed his
Dragonborn Vhakhun legions to capture Detrathesis. From there he processed to move north to conqueror
both Tel Shoku and Nirvidium. In a show of force, he used an allied true dragon to incinerate Tel Shoku, but 
completely spared Nirvidium. His main enemies where the Serovean High-Elves, and his human ancestors shared cultural
ties with the Mitosians. Nirvidium willingly surrendered, in turn their were given a great position of power in the empire
becoming the new capital. 

After the conquest, Rosinya would find itself as the heartland of an Empire spanning the northern half of Vilterra.
Over the next centuries it's human population exploded. Nirvidium expanded to back a metropolis rivaling the Elven 
cities of Serova and Junos, and by the great dragon war, was likely the most popularise city in the vilterra. Through
this population expansion, the Meneketes was able to field large armies, putting an immidate threat on the Serovean border.
The region would contribute to high tensions, both in the early Serovean-Meneketes wars and the proceeding cold war afterwards.

Meneketes rule undoubtably shaped the development of early rosinyean culture. As citizen of the empire, their spoke
Imperial Mitosi, a language that would later form the backbone of Common and Rosinyean. Their also adopted the Dragon cult 
of the half-dragons and dragonborn, primarly worshipping Thré and Nepaté. 

The region of Rosinya would also be significant during the Meneketes civil war. The human domianted cities of Nirvidium and
Detrathesis were key supporters of the Emperor, Scipio Meneketes, against the Dragonborn Vhakhun insurgencents. The Rosinyeans would 
also provide a valiant resitance against Serovean invasions during the Serovean-Meneketes Wars, with many of the 
most decessive battles happening in southern Rosinya. It is said that the orcs were only created to compete with 
southern Rosinyean fighters.

By the time of the Great Dragon War, Rosinya lost promiances as a major influence in the Meneketes Empire. The capital
was formally moved to Roanik (Gale), and the Contempary Emperors were increasingly influenced by Nordic culture. Yet 
None the less it semented itself as a force in the empire, and a target for elven aggression. 

\subsection{founding of Rosinya}
The \emph{Great Dragon War} rained hell upon the lands of the little rose. Farmland, forests, 
and grasslands were completely torn apart. The great cities of Nirvidium and Detrathesis where 
completed wiped off the face of vilterra. To the devastated survivors it seemed like their 
world had ended forever. However, like a phoenix, Rosinya rose from the ashes more powerful 
than ever.

\subsubsection{Aftermath of the War}
After the devastration of the Great Dragon War, Rosinya was left almost barren and depopulated. Refugees 
from Nirvidium and Detrathesis scattered and fought over the few fertile forests, rivers, and valleys that
remained after the war. Without the protection of the Meneketes, waves of monsters and orcish bandits flooded 
in the lands, terrorizing the local inhabitats. Only the Wood Elves benefited, as many tribes used the oppertunity 
to reclaim their tradtion lands. Still, even they didn't find much confort in the ruin leftover. 

The small fortress town of Braceo, remained a last bastion of civilization in the wasteland. It's inimportance
and strong defenses made it relativety untouched by Elven forces. It had also housed the Noble Pippin Chavele,
who was in line to rule the Province of Petrathesis. Remaining Meneketes Imperial forces rendezvous 
at the town, while survivors followed. Chavele declared himself the new lord of Petrathesis, but would keep his 
forces close in Braceo to protect himself and his wealth. 

In the rest of the region, independent groups set up small settlements and villages, defending each other from
constant monster attacks. This communities were self governoring, but still held on to their common Imperial 
identity. Many in desperation passed around prophecy, about the day in which their Emperor would return. 
The dangers of dust storms and the elements turned many Rosinyeans toward the worship of the minor god
Palid. Other groups, growing tight knit communities, turned to the worship of the bond fire deity, Hestiam.
Notable the first Templars of Palid ruled over one of these communities. And, on a fatefull summer day in 8867,
they would induct a disillusioned war veteran named João Inez. 

\subsubsection{João Agostino's Rebellion}
Only a decade after the Great Dragon War, war would once again break out in Rosinya. Pippin Chavele, having gone 
insane, declared himself the new Meneketes Emperor. He start securing communities around his neighborhood, at first 
being recieved with open arms. But upon resistances, his conquest became brutual. He attacked many communities 
just to increase his supply of slaves, and would reestablish the old Meneketes caste system seperating half-dragons descends,
Mitosians, Nords, Serofeans, and non humans. While these actions wouldn't have caused a stir in the old empire, years of 
people surivoring together changed these attitudes. Chavele would only become more radical, as distant news 
came of other leaders claim the title of Emperor. 

On May 22nd, 8887, João Inez, having been named \emph{João Agostino} by the Templars, staged a preasants protest
in Braceo. In responce Pippin Chavele publically tortured him. First cutting off half his right hand, then
blinding him, stab him, and hanging him on the city gates. His fellows templars the Orc, Yo'gru Un, and human
Martim Vanyado would pull his corpse down. However, much to their ammasement, he secretly survived, and
evidently without necromancy. Upon this miracle Joao Agostino was named leader of the templars.

Agostino despite his injuries would lead a successful guerrella campaign. He personally took part in
combat to inspire his own troops, strenghtening all of his other senses to become an efficentive and 
ferious warrior. It was said that with his prosethic bladed gaunlet hand and training with smite techniques,
he could kill any enemy in one punch. Even without personal prowess, he effectively commanded specialised 
warriors who used the wasteland to their advantage. With lucarious grasslands regrowing across Rosinya, 
Agostino also employed the extensive use of Cavalry giving him a major tatically advantage.
Many settlements flocked to his support, growing his own personal power.  

He took the oppurnity to build a new religion based off of Palid. He stress the importances of everyone 
including leadership were responsible to strict moral codes. He forbaded political corruption and talked 
down the sins of Greed, Envy, and pride. He outlawed slavery, the caste system, and noble prerecisists for 
positions of power. This slander caused Pippin Chavele to outlaw the worship of Palid, further isolating 
himself from the his people.

In desperation Pippin Chavele made a pact with a true dragon, hoping to gain a tatically advantage over
both the rebels and other imperial claimants. Upon hears this Agostino led a direct attack upon Braceo, 
succesfully leader his forces in a quick seige. He proceeded to form a party with Yo'gru Un, Vanyado, and
Balif, an allied monster hunter, to slay pippin's dragon. Agostino than tracked down Pippin and completely
incinerated him with a barrage of punches as to prevent him from being revived though spells or necromancy.

Upon this he declared, \emph{for now on the land of the Rosinya shall shake off the shackels Empire.
We only server one, and that's Palid}

\subsubsection{the great crusader}
Agostino immidately organized his new government. He sent messagers to all the major settlements,
employing them to sent representive to discusses their new governmental relationship. He also immediately 
set on rebuilding the region and secure fertile lands. At home in the city of Braceo, he immediately went 
about flexing his new position of power. Former dragon blood nobles where either executed or imprisioned,
quickly be replaced by Palid Templars and talented peasants. He also completely reorganized the city.

Being overpopulated, he sent Braceo preasants to colonize the neirby region. In this new settlements
Agostino would establish peasant led communes, creating the testing grounds for his new economic reforms. 

% (Thré and Nepaté become less popular and religions died out, their worship was associated with the old elite, and Palid and Hestiam where more apparentling)

In the years of 8891, would declare the start of the great pilgrimage. He sent messagers and settlers to the lands 
surrounding his new rosinyean state, with the intentions of converting more to the religion of Palid. As part of this
campaign, Agostino vastly expanded Rosinya's borders. In 8892, he sent forces and colonizists south taking a large 
sowth of area going down to the Anjoi river. By the end of the year his armies were within strike distance of Vara
and Junos. He ultimately stuck to taken advantage of the Anjoi's fertility but notably restrained from conquering the 
old elven cities. Many of his generals hated the high elves and saw it as a riteous crusade, and argued the Agostino
could supplant the Serovean Empire. However, Agostino disagreed, seeing the invasion as untatically sound, and pointless. 

Instead he refocused north, claiming the straights surronding the ruins of Nirvidium. His armies set up forts to protect 
the survivors of the city, who at that point were already fostering a new generation. Many settlement were already building 
on top of Nirvidium's foundation, so Agostino sent more settlers with the hopes of Reviving the anchient imperial capital. 

For four more years Agostino went uncontested, but in 8896, a new entity entered the region. For decades the minor noble,
Archimedes Citadil, had gone about creating a new imperial reminant in the inner sea. Ruling from modern Kamelon, Citadel 
would declare himself the legitimate successor to the Menekete's empire. His forces moved into the region around Nirvidium
aiming to gain legitimancy though controlling the formal imperial capital. Alarmed, Agostino would preemtively declare a
crusade to defend Nirvidium and Rosinya would Thre worshipping tryants. The templars score several immediate victories,
but ended up being bogged down by Citadil's forces. Agostino started to realize that Citadel was a respectable war general,
but still saw him as corrupt and a threat to his revolution. 

The war would go on for two years, until a Necromantic army attack both powers. Agostino ended the war seeing necromancers
as a significantly larger threat. Citadel and Agostino agreed to joint rule of Nirvidium, Citadel's conversion to Palid worship,
and Agostino stepping down from power (and Martim Vanyado taking over). The alliance fought a three year campaign against 
the necromancer king, eventually defeating him in 8901, but in the process losing Joao Agostino.  

% at some point mention who people wanted to follow Agostino's "chastety" but he was really just Asexual. 

\subsubsection{Saint Martim Vanyado}
Saint Agostino designated his most trusted apprentice Martim Vanyado to replace him.
Vanyado took power as soon as Agostino abduncated, but didn't truly start to use his 
great power, until after the death of his master. 

When news reached his eyes, Vanyado occastrated a grand funeral for Agostino. He used this
occasion to gather the new lords of Rosinya, and test their loyality to him. However, he also
rewritten the Rosinyean laws created by Agostino. This included rearranging the commune land 
grants, and granted more powers to his officials. He allowed commune leaders and lords to choose
their own successors. These changes signalled the transition to feudalism brought on by his reign.

%Vanyado was more moderate tan Agonstino except for hestiam followers, he was worse ended up banishing them 

%In 8911 Citadel's remiant collapsed into civil war and anarchy. Mostly being substanced by Citadel's chrasma. his son Aeos Citadel IV was bad 
% Vanyado conquerors and reincorruparates Nirvidium, now having been rebuilt into Nirvire

% 

\subsection{the invasions and reconquest}

\subsubsection{the prelude}
The year 8928 would mark a significant year on the continent of Vilterra. A united horde of dozens of Asakhani Tribes crossed
the mouth of the world into the formal territories of the Serovean Empire. Their leader Teljin 'the Wavemarker' sought to 
conquer the entire continent, which had still barely recovered from the Great Dragon War, 62 years before. However, for the 
kingdom of Rosinya it was only distance rumors. 

The aging Martim Vanyado was more focused on closer affairs. Their shakey allies, the Citadil's Remiant, had recently collapsed into civil war.
The Rosinyeans no longer trusting the remiant fully annexed Nirvidium, and sent templars north to prepare for another war with the Citadel. 
However, these anxiety melodied as by 8933 the Citadel's Remaniant completely collapsed into competing kingdoms. With the north
secured, it seemed like for a time Rosinya would have breathing room, but this peace would be short lived. 

In late 8937, the new Asakhan leader, Mahmud Ibn Teljin, secured a peaceful annexation of Junos and Vara. This put the
new empire right on Rosinya's border, sending a mixed reaction though the land. Many were ecsatic to witness the final
blow to high-elven rule, others were wary of these nomatic invaders. Martim was frantic in securing a diplomat mission to
the Asakhan. First he wanted to get in good terms with these new human rulers, second he hoped to convert Mahmud to 
the worship of Palid, and he finally wanted to secure Rosinyean lands on the Anjoi. 

It was late winter in early 8938, when the two leaders first met. They preformed extended talks in Junos for
much of the following year. Martim secured limited cooperation with Mahmud, insuring a temporary peace. 
Martim would allow free Asakhani passage through the Anjoi, while the Rosinyeans kept these lands. However,
Martim never secured any lasting relations, only aliening the Asakhan through his fervor. Either way,
way there was not much he could do. In 8942, Martim died of a stroke, leaving his regime in the hands of an 
unprepared ruling council. 

\subsubsection{the surge}
Everything quickly escalated in 8953. Mahmud Ibne Teljin is assassiated, leading to a two year power struggle 
between various elven and human factions in the Asakhan Empire. In 8955, Mahmud's 17 year old son Ahmet
barely secured the Asakhan throne. Within internal problem's surging through the empire, Ahmet wanted an
external victory to distract the country and secure his reign. Only a month into his reign, Ahmet launched
a surprise attack into Rosinya. 

The combined force of light Horse Archer, Wyverns, and Battlemages quickly overwelmed Rosinyean forces.
Within weeks, the Rosinyean's lost all territory along the Anjoi. Ahmet reoriented his forces to push
north, hoping to complete cripple Rosinya. Braceo surrended quickly, Ajetos and even the great Nirvire
fell within month, while most other intial Rosinyean forces were wiped out within half a year. 
Ahmet temporarily returned home to celebrate his victory. Remaining Rosinyean forces gathered in the 
in the last stronghold of Miridea. Preparing for a final stand in the war.  

In May 8956, a new wave of Asakhan forces entered Rosinya, with the intent to mop up any last resistance.
Ahmet emboldened by his previous victories, rushed to seige Miridea with the bulk of his new force. On 
May, 23st, 8956, Ahmet marched on Miridea, asking for surrender or death. 

According to one legend an old Rosinya, "who had probably seen the Dragon War," bites her thumb at the invaders. 
Asakhan forces, offended, cast fireball on the women. Only for her to surprisingly use counterspell. Her
illusion disappeared, revealing the figure of Palid. 

While this story is apocryphal, it reflects how strong rosinyean resistance was. Trained Asakhani horse
archers where wiped out by stapeding herds of bulls. Wyvern riders where massacred upon aeral driving 
runs, as insane peasant jumped upon their steads dragging them down into murdeous mobs. Powerful high-elven 
mages were challenged by holy templar knights. In all these chaos a peasant woman named Carla rose to a 
position of leadership. Through day and night, she spread the word of Palid, invigorating the masses of 
Miridea. At the gates of Miridea 400 northern berseckers stood strong. Day and night under the command of 
Carla they held against Ahmets rushed attacks. 

The quagmire of a seige lasted for nearly seven months, dragged Ahmets reputation into the mud. 
His lack of real battle tatics were fully demostrated in these battles. 
His previous victories, being more due to Asakhan strenght and numbers
were nullified. However, he wouldn't give into Rosinyean demands, his power and even life were on the land.

After several failed raids, Asakhani forces final broke into the city. Peasants hordes, templars, and royal 
bersecks fought to the last inch, as exhausted Asakhani forces massarced the city in vengences. Carla 
was brutually executed dying a matyr, however her efforts gave time for many important figures to escape. 
Most notably the lord of Braceo, Joao Miguelez, was able to flee north to modern Gale.

\subsubsection{the glorious return}
By 8956, Rosinya was under Asakhan occupation, but Ahmet's position was once again weak. His forces 
were humalated in Miridea, his reputation was shakey in Asakhan, while a pretender to Rosinya was 
still alive. Still needed to secure his own reign at home, and once again provide battle for his fighting
force, he embarked on a rash plan. The Asakhan invaded north and east of Rosinya into the lands of the 
former Citadel Remant. Thoughout the bay the once bickering kingdoms and the remaining Rosinya templars 
united to repell this incoming invasion. At the same time Joao Miguelez moved south to support a new 
key ally, the Kingdom of Camelon. 

In mid 8956, Allied Camelonian and Rosinyean forces repelled Asakhani forces in the Battle of Torel.
Later that year Armies led by the Templars would intrap major Asakhani forces in northern Camelonian
taking the oppurtnity to wipe them out. These battles would also allow the Rosinyeans to flex their 
new mastery of Pegasus and Wyvern riding. 

Simutanously, many who fled Rosinya years earlier would take to
the seas. Rosinyean pirates and privateers raided along the Asakhan coast, even stiking far south into 
the Avi and Aurabe. One of these pirate, Marina Calcio, would become infamous among the Asakhani. She
used a mysterious artifact to summon Dragon Turtles to battle.

In 8958, Rosinyean forces, Camelonian, and the Minor kingdoms mounted an offensive to retake Rosinya.
Their quickly overran Miridea, Sivid, and Nirvire, and official established the second kingdom of Rosinya.
for the moments being it looked like the Asakhan were on the retreat. However further allied advantaces 
were imdiately halted. 

Ahmet was assassinated, leaving his eight year old brother to the throne. However, in truth the general 
Khalid Ibne Mehmet held all power in the empire. With significant more battle and tatical expierence, he
more successfully pushed back Allied forces. In the battle of Ajetos, Joao Miguelez, lost all of his forces 
and was captured by Khalid. Later, Khalid would wipe out Camelon and Templar forces in the north planes 
near miridea. Having been exhausted by the war, the allies sued for peace.

Khalid gave Nirvire independence, allowing Camelonian nobles to rule it. He also allowed Joao Miguelez to 
retake his title, Lord of Braceo, but the lands of Rosinya proper would remain under Asakhan control. The 
allies were outraged, but were exhausted, conceding to a drawn out Asakhan victory. The Glorious return fails.

\subsubsection{the second war}
Both sides would rest for two decades. Khalid Ibne Mehmet would lose control of the Asakhan to the now
older Fahri Ibne Mahmud. Fahri started putting heavier restrictions on Rosinya, sercumventing local power
structures to increase Asakhan influence. Tension once again simmered, as the lords of Rosinya tired to 
contest Asakhan control over their area. Simutaeously, Rosinyean pirates and Camelonean forces had rebuilt 
hoping for a rematch. 

In June 8979, peasant riots broke out across the terriorties of Braceo, Vasquea (renamed Teyokir by the Asakhan),
and Ajetos. They were protesting increased taxes imposed by the Asakhan, although intially also targetting Joao 
Migelez de Braceo. The local lords along with the Asakhan crushed the peasants. But resentment was becoming more 
proment. Joao started to become more assertive about his position to the Asakhan, out of fear of more rebellions. 

Fahri started to take offensive to Braceo's descent, eventually sending Khalid, the now veteran war general. To help
resecure the provience. Khalid hoped to negotiate with Miguelez, but diplomacy broke down. Joao Miguelez started 
withholding taxes from the Asakhan, and would harass tax collects using his personal knights. Upon hearing this, 
Fahri called upon Khalid to stage a coup on Braceo. Relenctuantly, Khalid followed orders. 

In December, 8979, Khalid stormed into the city of Braceo. He quickly overwhelmed Miguelez's knights and arrested 
him. Khalid then installed another minor Rosinyean noble to power, hoping to quickly quel any dissicent. However,
the carrage transporting Lord Miguelez was intercepted by a group of thiefs. Intitally, the thief, were hoping to
rob from the lord. However, by sympathic to his plight, their released him. At the same time, more peasant riots 
broke out across the Asakhan provinces of Rosinya. Khalid, was forced on an untidely return to Serova, leaving the
puppet ruler of Braceo alone.

After a harsh winter, the territory of Rosinya was at the breaking point. Small scale revolts started beaking out,
Asakhan reinforces were ambushed, and several lords started declaring independence. In March 8980, former allies 
of Rosinya from two decade earlier, along with new players in the north meet up in Camelon. Among them was Connor 
Redfyr, a half-dragon descent of Aeos Meneketes, who'd just conquered the remain's of the Meneketes formal capital.
The new Queen, Lorraine Odela, of Camelon also played a promient roll within this meeting. This group would official
form the Second Coallition to restore Rosinya, and declared war on the Asakhan. Only a week after this Joao Miguelez 
recaptured Braceo, declaring the third kingdom of Rosinya. 

Fahri was fanatic in trying to crush this rebellion. He allowed Khalid to return to Rosinya, hoping to once again 
pull off a decisive victory. Before the generals arrival, coallition forces quickly recaptured Miridea, Ajetos, and
Sivid. Palid Templar crushed Asakhan defensive forces, meanwhile Nirvire wyvern and pegasus riders harassed enemies 
from the air. At sea privateering and raids continued as legendary pirate, Marina Calcio, mysteriously returned.

However, Rosinya victories were short lived. General Khalid returned, and immediately crushed coallations forces. 
Using superior battle field tatics, along with magic, he managed a continous string of victories. By the end of 
the year, all major Rosinyean cities were recaptured, and two-timed king Joao Miguelez once again had to flee. 

The war slowed. Khalid focused on defending Rosinya, but was constantly dealing with coallition forces. Decades ago, 
he opposed Ahmet's invasion of the far north, however he changed his mind. Seeing it as the only option to end the war,
Khalid captured Nirvire, and processed to move into Camelon. Once again he managered a string of decessive victories,
but could never capture Joao Miguelez. In anger he set Camelon aflame destroying everything in his path. 

Finally Khalid faced all the major coallition leaders in the Battle of Jor Dulee. Out numbered 4 to 1, 
Khalid still managered an impressive victory. Killing dozens of promient generals, capturing Queen Odela, and
effectively conquering half of Camelon. However, once again, Joao Miguelez escaped. Khalid continued on a year 
long chase to find the unifying leader. In the process, slowly losing his tatically edge. He kept winning in 
battle, but was taking heavier loses. Finally in 8982 he sued for peace.

Queen Odela seceded southern Camelon, and the Asakhan also annexed Nirvire. Miguelez claim on Braceo was revoked,
and royal Asakhani nobles were placed incharge of Rosinya. The Empire had only entrenched itself deeper north, and 
it was only a matter of time until the next war.

\subsubsection{the third and final of the major war}
By 8995, Joao Miguelez was still in hiding, long having lost his kingdom. He had become an old man, raising a secret
son Antonio to adulthood. General Khalid had taken up the position of Great Vizier once again gaining power over, 
King Fahri. Queen Odela, Redfyr's replacement Gale Stormcrown, and the Veteran, now Queen of Cossairs, Marina Calcio,
meet in secret to discuss a potential third coallition.

The northern kingdoms were left unstatisified with previous campaigns, and wanted revenge for the humalation by general
Khalid. At the same time, rumors of Miguelez's survival spread across the Adateira bay (bay next to vilterra). In 
Rosinya itself, many unground rebellions were formed to combat the Asakhan. This included the travelling Rose Guard. 
Founded by former Palid Templars, they spied on Asakhan nobles who their believed suberted Rosinyean laws to install
they own power. They undermined these lords by stealing their wealth, raiding their guards post, and insighting protests
among Rosinyean peasants. 

Vizier Khalid once again took interest in Rosinya. He knew, despite his best hopes, that another war was inevitable.
Through 8995 Khalid tansfered thousands of elite Asakhani cavarly and battlemage units into rosinya. He hand picked 
a young general, Telsin, to oversee defensive and counter-insurgency opporations in Rosinya. In response the Rose 
Guard grew it's efforts, and official contacted Marina Calcio for helping contest Asakhan power. 

Rosinyean pirates from across the Asakhan coastlines started raiding Asakhan merchant ships in an organized harrasment
campaign. The situation got worse enough, that the Asakhani upper council and king Fahri would approving giving 
more power to Vizier Khalid to deal with the crisis. Already by november 8995, The Asakhan were openly at war
on the sea. Only two months late, active combat started around Rosinya's major cities. The Rose Guard attacked 
Asakhan soldiers in open combat, and were suprisingly successful. Their new leader, Dorotheia, had managered to defeat
General Telsin during the battle of Braceo. In the process opening up the gateway for Braceo liberation from the Asakhan.
On Feburary 2nd, 8996, Dorotheia declared the Fourth Kingdom of Rosinya, installing a temporary Ruling council based on 
the political organization of Agostino's empire. 

This events sent shockwaves across vilterra. Vizier Khalid marched towards Rosinya, hoping to crush the last Rosinya
resistance once and for all. Adjucancently, northern kings rushed to support the uprising, haphazardly creating the 
third coallition. The Asakhan vizier smashed into Rosinyean with a vengeance. On land, the Rose guard lost most of their
gains, stuck to defending key cities in brutual seiges. At sea, Asakhani wizards reverse engined Calcio's monster summoner
trick. They unleased a Kraken upon the unsuspecting Rosinyean pirate, quickly regaining the coastal front. Queen Odela's
forces in Camelon quickly got bogged down retaking southern Camelon, and coundn't led any aid to Rosinya.

In early 8997, after a year of fighting, it seamed as if Rosinya was in it's darkest hour. Dorotheia was forced to flee 
to Nirvire, Queen Odela had only barely stratched Asakhan defenses in Camelon, and Khalid despite his advancing age once
again provided his tatical might. However, on April 7th, the Asakhani Kraken was defeated by a daring pirate crew. A crew
which included former king Joao Miguelez and his son Antonio (Their probably weren't involved fighting the Kraken,
yet the defeat become heavily associated with the royal dynasty, hints why they have a kraken as their symbol) The news 
spread fast, and Vizier Khalid was furious. His desire to end a decades long fued, overcame his logical thought process,
as he took direct control of naval opporations to capture Miguelez. 

Khalid assembiled a massive fleat at lightning speed and set sail. On May 17th, he interspected Rosinyean forces, engaging
in the most concicential battle of the war. Both sides surived termendious casualities, and partically destroyed both of 
those feats. In the end Khalid would finally get his wish, but at a great cost. Mid battle, his personal wyvern, caught, 
killed, and ate Joao Miguelez, but in the process leaving himself vulnerable to counter attack. Khalid was captured by 
Calcio, and executed by burning. Calcio, sent in envoy to Dorotheia, declaring their new king Antonio. 

The Asakhani were shocked by the death of their greatest military general and leader. Because of their numbers and
superior technology, their were able to drag the war out for three more years. Yet, with a new Rosinyean king, the 
alliance in the north, now had far more morale and unity then ever. Camelon was reunited, northern Rosinyea was
liberated, and Asakhan faced intense stagering loses. The front settled along the Viere river, when King Antonio 
(now Antonio de Braceo) signed a piece treaty with the Asakhan upper council. The battle lines were submitted as final
borders, and Antonio was given control of Rosinyean, with the exception of Nirvire which was to be it's own kingdom.

In the year 9000, Rosinya final summented lasting independence. 

\subsubsection{Beginning of Braceo rule and the other wars}
The end of the third Rosinyean-Asakhan war only secured indepence of northern Rosinyea. Thoughout the 91st century,
both the Rosinyeans and Asakhan had three more wars. however, these conflicts aren't oft as remembered. The Asakhan
and alliance in the north also ended up in stalemates, and both sides became tired of fighting. Were once kings like 
Nirvire saw the Asakhan as enemies who opposed their fellow Rosinyeans and Vasq people, their now saw them as
luciative trading partners. The Dwarven -> Gale -> Nirvire -> Asakhan trade route became highly profitable. 
Rosinyean's allies quickly abanoned them, quickly making future wars lopsided towards the Asakhan's favor.

At the same time, the Asakhan wanting to keep their new trade partner didn't dare annexing Rosinya again. This 
created a cycle in which Rosinya would fight to gain back land, only for the Asakhan to resubmit the statis quo. 
The Rosinyean's slowly gave up on retaking Vasqea. Their stopped preforming all out wars, and focused on naval
proxy wars to disrupt Asakhan trade. These would by known as the Pirate Wars lasting for decade. However, this 
only served to anger their previous allies, causing the Nirvire to join the Asakhan in the war.

The Piracy Treaty of 9257, officially ended all directed hostilties between the Rosinyeans and Asakhan. This submit
Rosinyean new submissive role in the proceed two centuries of Asakhan dominantion. The Braceo kings afterwards tried 
reorganizing Rosinya to fit and thrive within this new world order. However underlying resentment surived the centuries.

While the later Rosinyean-Asakhan conflicts were happening, Rosinya itself started to see massive change. In 
the aftermath of the third Rosinyean-Asakhan war, Antonio de Braceo I, rushed to submit himself as the leader 
of Rosinya. In 9001, he married Dorotheia, and disbanded the Rose Guard to return control of Rosinya back to
the monarchy. He proceeded to reinstall various exelled lords to power, while set up new royal dynasty through
land grant to rich preasants. He allowed regional lords within his kingdom to field their own knights and militia
with the idea that each of them would be able to defend themselves in war. 

Despite being deticated to the feudal system, Antonio allowed his action to get mediated by the existing ruling 
council. Inspired by the council's of Agostino's first Rosinyean kingdom, the ruling council allowed for limited
representation from peasants and smaller local leaders. This council evolved into the Rosinyean Cortes, a 
limited democratic system which lasted until the War of the Witch King. Antonio's rule helped establish the system
of government that would domiant Rosinyea for nearly four centuries. 

% Rosinya more feudal under Braceo. After Piracy treaty their become Capitalist and open to foreign trade


% In 8922, the Asakhan under Telyin enter Vilterra. In 8930, Mahmud finishing conquest of the Seroveans. In 8945, Rosinya is conquered by the Asakhan
% the northern kingdom invigorated by Saint Carla of Rosinya led a decades long resistances. By 8997, Rosinya is mostly restored, although Agostino's
% anjoi conquest remain in Asakhan hands, Rosinye living their become the Vasq people. Joao Braceo I claiming descent from Martim Vanyado becoming 
% the first king of the current dynasty. Also palid vikings of the Suewel supported the defense of Rosinya. 

\subsection{the contempary era}
Asakhani domaination of the world significantly impacted the development of the Rosinyean start. The Piracy 
Treaty of 9257, official ended centuries of Asakhan-Rosinyean conflicts. The then king at the time, 
Tomas de Braceo I, started a transition into loose Asakhan cooperation. The time period saw the shrinking of
monarch power, federalization, and the growth of Rosinyean middle class. This would be the status quo for two
centuries during the pax Asakhan.

However, appoarching into the modern day, the Rosinyeans puhed back against the Asakhan. The 94th century saw
the start of Rosinyeans oversea's colonization, along with great technological advancement. The Rosinyean got
rich, gaining more influence of vilterrean trade and greater power projection. New guilds, born out of Rosinya's 
middle class, wanted to replace the old aristocrats. Their allied with the king to submit more power and centralize
the government. 

In the meantime, Asakhani aggression, intervenetion, and foreign mishaps, alienated their once former trade
patterns. To the Dwarves and northern human kingdoms, the Rosinyeans looked like the good guys, despite their
horrifying treatment of non-humans. The War of the Witch King was the straw that broke the camels back. An
event in where the Rosinyeans submitted themselves on the right side of history.

\subsubsection{Rosinya entering the Pax Asakhan}
King Tomas de Braceo I, was well known for signing the 9257 Piracy Treaty, however this wasn't necessarily 
a strict turning point within his reign. For decades Asakhani-Rosinyean relations had been warming up,
and both sides were edge towards peaceful diplomatic relations. In 9251, Tomas, become the first 
Rosinyean leader in centuries to travel to Serova. Along with members of Camelon's Odela, and Nirvire's 
Ducal dynasties, Tomas participated in a diplomatic convenfence establish by the Asakhan. It was a very 
unique expirementation for the world, and for many a sign of great hope. 

This first convenfence didn't really accomplish anything, but would help establish the route of Rosinya 
participating in diplomatic relations. More monumamental was the 9253 treaty of Serova, in which King Tomas
official relicished all claims to Vasqea. In term the Great Vizier, Abdulla Hassamon, promised granteed independence 
for Rosinya. The 9257 treaty, was only the final cherry on top, for the slow deacelation of conflict. 

Soon Tomas tunred around to restructure Rosinya for this new age. He would scale back trade restriction
with the Asakhan, and laxed border security. In the 9260s, he'd introduce Common as a secondary 
lanuage in many facates of life, as it was more familiar to Asakhan trades then Rosinyean. 

Tomas' daughter, officially crown Queen Claudia de Braceo would only further appifly those existing trends.
She made learning Common as a second language mandatory for governmental officials and military generals,
allowed Asakhani citizens free travel in Rosinyean territory, and halfed most trade tariffs. Along with 
this outward diplomatic motions, she made parallel reforms to help change Rosinya itself. 

At the time of Claudia's rule, Rosinya already had a growing merchant class, which only benefited from
getting intergation with the rest of the world's economy. Claudia activaly funding many merchant 
adventures, and cooperated alongside them on economic issues. She also invested heavily into the 
expanding city of Porto Viere, which had grown due to Asakhani connection. 

Founded as a fishing settlement during Asakhani occupation, Porto Viere rapidly expanded during the 
23rd century. It grew from backwater to the most properous hub for business and involvation. This 
city became the birth place for major guilds, which quickly expanded their operations outside to other
major cities. Claudia would only further help the cities momumental growth, and by the end of her reign
in 9311, Porto Viere would become the richest and most populous city in Rosinya. 

Claudia's support for economic ventures extended to overseas expenditions. In 9281, she invested into
an explore named Rico. His original mission was to find a path to the holy lands of Kori No Azumi.
With according to his reports were successful. however, on his return journey he ended up stumbling
upon a mysterious land mass southeast of vilterra and north of Seforea. Returning from his journey,
his description of the contient led many to believe it was the lost land of Seyaka'ir recorded by 
anchient elven explores. Throughout the last decades of her reign, Claudia continued to send explores
to chart out Seyaka'ir for its natural riches. 

The queen was monumental in advancing Rosinyean towards a more modern era, but it turn many of her 
desicions would lay the ground work for future conflict. Little to her knowledge would Rosinya turn
into an oversees empire, nor would her contabutions to the guild, make them extremely powerful.

\subsubsection{Onto to the modern era}
In 9324, a Rosinyean trade settlement in Seyaka'ir struck sliver. And only a year later the first 
shipment arrived in Porto Viere. Celebration come along the city streets as this daring expendition
was successful. Elsewhere however, other major Vilterrean powers become nervious. Sliver wasn't 
the start of Asakhan and Dwarven worries about the rising country. 

During the late reign of Queen Claudia, Rosinya found that produce such as coffee, sugar cane, and 
tabacco grew extremely well in the tropical climate Seyaka'ir. The queen allowed Porto Viere based
guilds to transport prisioners to Seyaka'ir to work for labor on newly established plantations. 
This prisioners were disporportiately Wood Elven rebels, a fact that would help for the later
reintroduction of Rosinyean slavery. 

This cash crops gave Rosinya a competitive edge over all in the coastal Vilterrean markers, making 
the country quite rich. More and more Dwarvish and Nirvire gold start flowing into Rosinya a fact 
which scared Asakhani leadership. In 9316, the Asakhan and Rosinya's struck a temporary deal where
other countries could freely trade with Rosinya's colonies. However, this deal was very one sided,
as the Rosinya control the ocean, giving them easy leverage over Asakhani vessels. 

Increased colonial trade eventually led to a Rosinya naval build up with stuck fear in the Asakhan. 
In 9322, the Asakhan and Rosinyeans signed a second treaty, limited Rosinyean naval size, in
exchange for Asakhani help crack down on their own domenstat pirates..

So when the Rosinyeans struck silver. It sent a shock wave of distress to many rulers of Vilterra.
The Dwarves were particularly concerned as they secure monolopy on metal was being threatened. 
At the same time many economic analysises feared the reduction in sliver's value which could led
to a market crash. Even beyond that, silver gave Rosinya the raw material to create their own 
magic items, giving them a new military advenure against the Asakhan. 

This new found tension cumulated until the 9326 Nirvire bay incident. A major Rosinyean guild 
cargo ship, was seiged by a wizard cultist, leading to a drawn out naval engagement. Following 
this, evidence was found linking this mage to Thunderpike dwarven merchenaries. Immediately 
many Rosinyean blamed the Asakhan for inciting the incident, and fear of another war was on
the horizon. However, cooler heads prevailed. 

For the most part the Nobility and Guilds of Rosinya were not interested in war. The Asakhan 
had become their largest trading partner, and Rosinya didn't have any allies to substane a 
war. The Rosinyean King, Great Vizier, Thunderpike king, along with several Guild leaders 
meeting up to discuss a new treaty to target economic woos. In the 9327 Nirvire treaty 
Rosinya agreed to regulate of Silver, in exchange for Asakhan and Dwarven recognization 
of oversees territories. At the same time, the dwarves, put a Rosinyean guild member on the 
council of hundred kings, so that the dwarves could regulate Rosinyean silver.

Many at the time hoped that trade intergration would prevent Rosinyean aggression. The 
Dwarves particularly saw this is a golden opportunity, making the Rosinya's part of their
cartel instead of rivals. A move that would help their relationship, and contribute to 
the current shift of Dwarven support away from the Asakhan and to the Rosinyeans. Even 
still this move wouldn't completely buy piece on the continent.

\subsubsection{Colonization of Seyaka'ir}
Oversea, the Rosinya had a whole other set of political challenged face. The land of 
Seyaka'ir was populated. It was primarly split between three groups. Like Vilterra,
Seyaka'ir had scattered wood elves tribes, having lived on the continent for as long 
as remebered. Their shared their lands with two umbrella human groups. From the north
were the people the Roinsyean called Tahão, and from the south were migrate groups
from Seforea. Even within these broad distuiguished there were hundreds of independent
ethnic groups, kingdoms, tribe, city states, all competing for power in the region.

Rosinyean guilds carefully exploited this portugal situation for their own gain.
They carefully chose easily manupulated allies, and pitied rival kingdom against 
each other. In the end carefully grabbing land to expand plantations and securing 
important national resources. The wars also saw them gain hold of wood elven war 
prisions, which their then set to work cash crop plantations. The colonies would
see the first time that the Rosinya's would exemplate non-humans from protections 
against enslavement, a trend that only continuned across the century. And in 9349,
the Rose Thorn Brotherhood was formed, eventually becoming the Slave Traders Guild. 

Human indegenous groups weren't trended well either. Many were forced out of their 
land, and inoccent villages were desimated by raids. Rosinyean trade of magic items,
cannons, and early construct technology heated competition between Seyaka'ir groups 
leading to the increase in brutual conflict across the continent. 

All this chaos made Rosinyean guild rich, yet their still had a heavy cost to all
this conquest. The indigenous peoples of Seyaka'ir were still quite strong in 
battle, and the depths of the jungles contained unimaginable monsters. The Sugar, 
tabacco, and coffee guilds had to grow personal militias to protect their claims 
in Seyaka'ir. This process was aided by the rise of the artificier's guild (also
known as the mechanic guild), which proceed constructs that the Rosinyean could 
use in combat. 

This growth in guild wealth, resources, and even military power would soon return 
to mainline Rosinya. Overtime Guild leaders grew closer with the monarch, recieving
more direct aid. Within a century they had suppressed the power of the old nobility,
coming to governor Rosinyean life.  

\subsubsection{ }

% the Asakhan and Dwarves recognized Dros Jastar's Gale to sequire gold trade routes.
% the Rosinyeans were the first to aid Sean Blackcrow and his knights against Dros Jastar
% Camelon would soon follow Rosinya in joining the war
% General Mustafa Demir of the Asakhan was always pro Blackcrow, but it wasn't until later in the war that he convinced Vizier Ahmet Yasin to switch sides
% Nirvire remained neutral and still traded with Dros Jastar, as their economy was dependent on trade, this made the leader Marie Ducal extremely unpopular with
%   many Nirvirean's supporting Rosinyean annexation
% Having picked the wrong side of the war, and humalated by Sandal, Ahmet Yasin was forced to resign. Mustafa Demir took over as Great Vizier. 
% King Pedro de Braceo II, would disolve the cortes during the War of the Witch King. Late leading to the first rebellion within Rosinya. Rebellion inevitable failed 



\end{document}
