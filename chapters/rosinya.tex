\documentclass[../main.tex]{subfiles} 

\begin{document}

%%%%%%%%%%%%%%%%%%%%%%%%%%%%%%%%%%%%%%%%%%%%%%%%%%%%%%%%%%%%%%%%%%%%%%%
%                               Intro                                 %
%%%%%%%%%%%%%%%%%%%%%%%%%%%%%%%%%%%%%%%%%%%%%%%%%%%%%%%%%%%%%%%%%%%%%%%

\emph{Rosinya}. A old kingdom, now growing into a grand empire. A land that is rapidly tranforming, 
at the cusp of a great change that will some consume the world. 
Where many may see opportunity, others only find the schackles of an oppressive regime.

What ever may occur in this most holy of domain, the whole world watches. 


%%%%%%%%%%%%%%%%%%%%%%%%%%%%%%%%%%%%%%%%%%%%%%%%%%%%%%%%%%%%%%%%%%%%%%%
%                               Geography                             %
%%%%%%%%%%%%%%%%%%%%%%%%%%%%%%%%%%%%%%%%%%%%%%%%%%%%%%%%%%%%%%%%%%%%%%%

\section{Geography} 
Sitting right on the center of vilterra's east coast. Rosinya enjoys a warm meditarean climate
with a rich land and coastal ecosystem. The country spreads across a diverse geographic
region which includes grasslands, temperate forests, low mountains, and ria coastlines.

\subsection{Weather} 
It is mostly sunny thoughout the years, exceptable for rain in the winter. 
This hosiptal environment is oft attributed to the grace of the revered 
storm god, \emph{Palid}. Although their might be an underlying current of his wrath,
as southern Rosinya has some of the roughest coastlines on the continent.
This tough enviroment has bred a powerful Naval culture which contributes 
to Rosinya's ability to project power overseas. 


%%%%%%%%%%%%%%%%%%%%%%%%%%%%%%%%%%%%%%%%%%%%%%%%%%%%%%%%%%%%%%%%%%%%%%%
%                               Ecology                               %
%%%%%%%%%%%%%%%%%%%%%%%%%%%%%%%%%%%%%%%%%%%%%%%%%%%%%%%%%%%%%%%%%%%%%%%

\section{Ecology}
%INSERT SOMETHING

\subsection{Fauna and Flora}

\subsubsection{the farmlands}
Due to rosinya's unmatch fertile soil, agricultural field prolifate the entire 
kingdom. These fields are dominanted by grains such as wheat, barely, oats and vegetables
such as cabbage, kale, and lettuce. However, Rosinya is also a major producer, 
consumer, and exporter of rice. 
(in fact exporting rice to the dwarves, has become a compitition between Rosinya and the Asakhan).

Agricultural land isn't just reserved for crops, as Ranching is also domiant in Rosinya.
Cows, chickens, and pigs are by far the most common farm animals. Cheese, milk, and Egg
based products are a long time stable in Rosinyean dishes. On special occassions pigs or
cows are slaughtered to provide the indegredents for steak and sausages.

In the past, farmlands fell victim to monster targeting livestock. However both independent 
monster hunter groups and an, increasingly interventitist, central government, have managered 
to stave these attacks.

\subsubsection{fishing and sea life}
Due to the abdunce of editable sea life, Rosinya has grown a large fishing indrustry.
Upon the stalls in major cities of Rosinya it isn't uncommon to find a diverse 
supply of seafood, including fish, squid, clams, and even octopus. Although not all
of these fish are avalible to everyone. Only the rich merchants, guild members, and nobles
enjoy such cuisine as octopus. Meanwhile, the average worker often eats markerel, cod, 
or sardines. This aquatic food culture even mades it deep into the countryside. Farmers would 
rathers eat the tuna caught downstream, than slaughter a cow that could otherwise 
provide valuable milk and cheese.

\subsection{Hostile Creatures}
Despite the best efforts of the Rosinyean government and independent monster hunters.
Many mysterious creatures still roam the deep woods of Rosinya.   

\subsubsection{subnatural phenomania}
For some inexplicable reason, in the west reaches of Rosinya undead and fiendish 
creatures occasional return to the mortal plane during the night. Their prowel around, 
some searching from victims to drag prematurely into the afterlife, others more 
innocenly, observe the joys of mortal life their now lack. 

\subsubsection{horror below and above the sea}
Well at the very least the kingdom can fight off fiends and necromany on land, the 
seas along Rosinya's coasts are a whole different story. Beneath the waves the 
demonic merrow, krakens, sea turtles, and the sort lurk, waiting to strike on unsuspecting 
victims sailing on the surface. There are many a tale of lonely fishermen being
dragged down by sirens, or travelling families being petrafied by swarming cockatrice.

On the rare occasion a hydra or sea turtle graces the shores, drawing the full
attention of Rosinya's mighty navy. 

With the advent of the caravel and blackpowder cannons, naval patrols have become more 
successful at challenging these threats. Yet, there is alway's air of caution 
when heading out to sea.

% \section{demographics} 

\section{internal politics}
Rosinya is a land expirencing rapid and unparalleled change. This is most evident within
it's mutanting govermental system and tumultuous domenstic politics.

\subsection{expanding monarchary}
The position of \emph{Rei Dom}, the lord king, has grown increasing more powerful. 
The current ruler \emph{João Braceo IV} and his precessors have taken steps to
centeralize the government and increase their own political power. More then ever, 
the monarch intevenes in local affairs, and has stunted the authority of the noblity.
The most drastic of these actions happen 18 years ago, when the former king \emph{Pedro Braceo II}
desolved the \emph{Cortes}.

This power shift has been aided by the establishment of the \emph{Inquistorius} and the 
backing of major Guilds. This rise of power has severally split the country. Many praise the 
strenghting of Rosinya due to the Kings divine guidance, while others have thrown 
accusations of tyranny.

% \section{geopolitics}


%%%%%%%%%%%%%%%%%%%%%%%%%%%%%%%%%%%%%%%%%%%%%%%%%%%%%%%%%%%%%%%%%%%%%%%
%                               history                               %
%%%%%%%%%%%%%%%%%%%%%%%%%%%%%%%%%%%%%%%%%%%%%%%%%%%%%%%%%%%%%%%%%%%%%%%

\section{history} 
Even the old Kingdom of Rosinya, is relatively young on the scale of Vilterra's anchient 
history. Yet the land itself has lived though it all.

\subsection{anchient history}
Like the rest of the continent, the lands now known by Rosinya were colonized by the 
Serovean Empire during the conquests of An Sa'ora. No records exist before this and even
records thoughout the rest of the Serovean's millennia reign are sparse. It is known that 
Wood Elven and Gnome tribes have always inhabitated forested parts of the region. High Elves
intially stuck to grandiose coastal cities, but would venture inland to magically terraform.
In fact the abducance of flowers in the region is likely due to Serovean engineering. 

The Serovean city of \emph{Tel Shoku} directly south of modern Nirviré, was by far the most 
populus settlement. It would serve much the same purpose of modern Nirviré; being a trade choke 
point between the north and south of Vilterra. 

\subsubsection{first human settlement}
The first humans to arrive in modern Rosinya were Nordic. They arrived a millinium and a half ago 
to escape famine caused by a mini ice age. At the time the Serovean's had collapsed into puesdo-anarchy
so human settlement remained relatively undisturbed. These groups would only form small clans, preforming 
raids against both Wood Elves and High Elves for resources.

It wouldn't be for two centuries until humans from the Archipelligo of Mitos arrived.
The Mitosian settlers were much more organized, bringing fleets of ships with the materials 
to form permeant cities. They also brought professial soldiers and were more knowledgable in the ways of 
magic. Detrathesis was the first of this settlements founded knew the modern city of Ajetos. Not long afterwards
the legendary naval general, Sirio, would found the settlement of Nirvidium, direct precussor of modern Nirviré.

To secure their new territory, Mitosian city states, allied themselves with the Serovean Empire. The Serovean High
Elves saw them as more Civilized then other human groups. They grant the Mitosians land, in exchange for doing 
the dirty work of repelling Nordic and Wood Elven incusions. Nirvidium and Detrathesis abused this relationship
to control the region, and insure future human domination.  

\subsubsection{The Meneketes Empire}
The year 8268 saw a monumental shift for the entirety of Vilterra. The Half-human Half-dragon king, Aeos Meneketes,
embarked on a continent spaning conquest, and Rosinya was his first target. In April 8268, he employed his
Dragonborn Vhakhun legions to capture Detrathesis. From there he processed to move north to conqueror
both Tel Shoku and Nirvidium. In a show of force, he used an allied true dragon to incinerate Tel Shoku, but 
completely spared Nirvidium. His main enemies where the Serovean High-Elves, and his human ancestors shared cultural
ties with the Mitosians. Nirvidium willingly surrendered, in turn their were given a great position of power in the empire
becoming the new capital. 

After the conquest, Rosinya would find itself as the heartland of an Empire spanning the northern half of Vilterra.
Over the next centuries it's human population exploded. Nirvidium expanded to back a metropolis rivaling the Elven 
cities of Serova and Junos, and by the great dragon war, was likely the most popularise city in the vilterra. Through
this population expansion, the Meneketes was able to field large armies, putting an immidate threat on the Serovean border.
The region would contribute to high tensions, both in the early Serovean-Meneketes wars and the proceeding cold war afterwards.

Meneketes rule undoubtably shaped the development of early rosinyean culture. As citizen of the empire, their spoke
Imperial Mitosi, a language that would later form the backbone of Common and Rosinyean. Their also adopted the Dragon cult 
of the half-dragons and dragonborn, primarly worshipping Thré and Nepaté. 

The region of Rosinya would also be significant during the Meneketes civil war. The human domianted cities of Nirvidium and
Detrathesis were key supporters of the Emperor, Scipio Meneketes, against the Dragonborn Vhakhun insurgencents. The Rosinyeans would 
also provide a valiant resitance against Serovean invasions during the Serovean-Meneketes Wars, with many of the 
most decessive battles happening in southern Rosinya. It is said that the orcs were only created to compete with 
southern Rosinyean fighters.

By the time of the Great Dragon War, Rosinya lost promiances as a major influence in the Meneketes Empire. The capital
was formally moved to Roanik (Gale), and the Contempary Emperors were increasingly influenced by Nordic culture. Yet 
None the less it semented itself as a force in the empire, and a target for elven aggression. 

\subsection{founding of Rosinya}
The \emph{Great Dragon War} rained hell upon the lands of the little rose. Farmland, forests, 
and grasslands were completely torn apart. The great cities of Nirvidium and Detrathesis where 
completed wiped off the face of vilterra. To the devastated survivors it seemed like their 
world had ended forever. However, like a phoenix, Rosinya rose from the ashes more powerful 
than ever.

\subsubsection{Aftermath of the War}
After the devastration of the Great Dragon War, Rosinya was left almost barren and depopulated. Refugees 
from Nirvidium and Detrathesis scattered and fought over the few fertile forests, rivers, and valleys that
remained after the war. Without the protection of the Meneketes, waves of monsters and orcish bandits flooded 
in the lands, terrorizing the local inhabitats. Only the Wood Elves benefited, as many tribes used the oppertunity 
to reclaim their tradtion lands. Still, even they didn't find much confort in the ruin leftover. 

The small fortress town of Braceo, remained a last bastion of civilization in the wasteland. It's inimportance
and strong defenses made it relativety untouched by Elven forces. It had also housed the Noble Pippin Chavele,
who was in line to rule the Province of Petrathesis. Remaining Meneketes Imperial forces rendezvous 
at the town, while survivors followed. Chavele declared himself the new lord of Petrathesis, but would keep his 
forces close in Braceo to protect himself and his wealth. 

In the rest of the region, independent groups set up small settlements and villages, defending each other from
constant monster attacks. This communities were self governoring, but still held on to their common Imperial 
identity. Many in desperation passed around prophecy, about the day in which their Emperor would return. 
The dangers of dust storms and the elements turned many Rosinyeans toward the worship of the minor god
Palid. Other groups, growing tight knit communities, turned to the worship of the bond fire deity, Hestiam.
Notable the first Templars of Palid ruled over one of these communities. And, on a fatefull summer day in 8867,
they would induct a disillusioned war veteran named João Inez. 

\subsubsection{João Agostino's Rebellion}
Only a decade after the Great Dragon War, war would once again break out in Rosinya. Pippin Chavele, having gone 
insane, declared himself the new Meneketes Emperor. He start securing communities around his neighborhood, at first 
being recieved with open arms. But upon resistances, his conquest became brutual. He attacked many communities 
just to increase his supply of slaves, and would reestablish the old Meneketes caste system seperating half-dragons descends,
Mitosians, Nords, Serofeans, and non humans. While these actions wouldn't have caused a stir in the old empire, years of 
people surivoring together changed these attitudes. Chavele would only become more radical, as distant news 
came of other leaders claim the title of Emperor. 

On May 22nd, 8887, João Inez, having been named \emph{João Agostino} by the Templars, staged a preasants protest
in Braceo. In responce Pippin Chavele publically tortured him. First cutting off half his right hand, then
blinding him, stab him, and hanging him on the city gates. His fellows templars the Orc, Yo'gru Un, and human
Martim Vanyado would pull his corpse down. However, much to their ammasement, he secretly survived, and
evidently without necromancy. Upon this miracle Joao Agostino was named leader of the templars.

Agostino despite his injuries would lead a successful guerrella campaign. He personally took part in
combat to inspire his own troops, strenghtening all of his other senses to become an efficentive and 
ferious warrior. It was said that with his prosethic bladed gaunlet hand and training with smite techniques,
he could kill any enemy in one punch. Even without personal prowess, he effectively commanded specialised 
warriors who used the wasteland to their advantage. With lucarious grasslands regrowing across Rosinya, 
Agostino also employed the extensive use of Cavalry giving him a major tatically advantage.
Many settlements flocked to his support, growing his own personal power.  

He took the oppurnity to build a new religion based off of Palid. He stress the importances of everyone 
including leadership were responsible to strict moral codes. He forbaded political corruption and talked 
down the sins of Greed, Envy, and pride. He outlawed slavery, the caste system, and noble prerecisists for 
positions of power. This slander caused Pippin Chavele to outlaw the worship of Palid, further isolating 
himself from the his people.

In desperation Pippin Chavele made a pact with a true dragon, hoping to gain a tatically advantage over
both the rebels and other imperial claimants. Upon hears this Agostino led a direct attack upon Braceo, 
succesfully leader his forces in a quick seige. He proceeded to form a party with Yo'gru Un, Vanyado, and
Balif, an allied monster hunter, to slay pippin's dragon. Agostino than tracked down Pippin and completely
incinerated him with a barrage of punches as to prevent him from being revived though spells or necromancy.

Upon this he declared, \emph{for now on the land of the Rosinya shall shake off the shackels Empire.
We only server one, and that's Palid}

\subsubsection{the great crusader}
Agostino immidately organized his new government. He sent messagers to all the major settlements,
employing them to sent representive to discusses their new governmental relationship. He also immediately 
set on rebuilding the region and secure fertile lands. At home in the city of Braceo, he immediately went 
about flexing his new position of power. Former dragon blood nobles where either executed or imprisioned,
quickly be replaced by Palid Templars and talented peasants. He also completely reorganized the city.

Being overpopulated, he sent Braceo preasants to colonize the neirby region. In this new settlements
Agostino would establish peasant led communes, creating the testing grounds for his new economic reforms. 

% (Thré and Nepaté become less popular and religions died out, their worship was associated with the old elite, and Palid and Hestiam where more apparentling)

In the years of 8891, would declare the start of the great pilgrimage. He sent messagers and settlers to the lands 
surrounding his new rosinyean state, with the intentions of converting more to the religion of Palid. As part of this
campaign, Agostino vastly expanded Rosinya's borders. In 8892, he sent forces and colonizists south taking a large 
sowth of area going down to the Anjoi river. By the end of the year his armies were within strike distance of Vara
and Junos. He ultimately stuck to taken advantage of the Anjoi's fertility but notably restrained from conquering the 
old elven cities. Many of his generals hated the high elves and saw it as a riteous crusade, and argued the Agostino
could supplant the Serovean Empire. However, Agostino disagreed, seeing the invasion as untatically sound, and pointless. 

Instead he refocused north, claiming the straights surronding the ruins of Nirvidium. His armies set up forts to protect 
the survivors of the city, who at that point were already fostering a new generation. Many settlement were already building 
on top of Nirvidium's foundation, so Agostino sent more settlers with the hopes of Reviving the anchient imperial capital. 

For four more years Agostino went uncontested, but in 8896, a new entity entered the region. For decades the minor noble,
Archimedes Citadil, had gone about creating a new imperial reminant in the inner sea. Ruling from modern Kamelon, Citadel 
would declare himself the legitimate successor to the Menekete's empire. His forces moved into the region around Nirvidium
aiming to gain legitimancy though controlling the formal imperial capital. Alarmed, Agostino would preemtively declare a
crusade to defend Nirvidium and Rosinya would Thre worshipping tryants. The templars score several immediate victories,
but ended up being bogged down by Citadil's forces. Agostino started to realize that Citadel was a respectable war general,
but still saw him as corrupt and a threat to his revolution. 

The war would go on for two years, until a Necromantic army attack both powers. Agostino ended the war seeing necromancers
as a significantly larger threat. Citadel and Agostino agreed to joint rule of Nirvidium, Citadel's conversion to Palid worship,
and Agostino stepping down from power (and Martim Vanyado taking over). The alliance fought a three year campaign against 
the necromancer king, eventually defeating him in 8901, but in the process losing Joao Agostino.  

% at some point mention who people wanted to follow Agostino's "chastety" but he was really just Asexual. 

\subsubsection{Saint Martim Vanyado}
Saint Agostino designated his most trusted apprentice Martim Vanyado to replace him.
Vanyado took power as soon as Agostino abduncated, but didn't truly start to use his 
great power, until after the death of his master. 

When news reached his eyes, Vanyado occastrated a grand funeral for Agostino. He used this
occasion to gather the new lords of Rosinya, and test their loyality to him. However, he also
rewritten the Rosinyean laws created by Agostino. This included rearranging the commune land 
grants, and granted more powers to his officials. He allowed commune leaders and lords to choose
their own successors. These changes signalled the transition to feudalism brought on by his reign.

%Vanyado was more moderate tan Agonstino except for hestiam followers, he was worse ended up banishing them 

%In 8911 Citadel's remiant collapsed into civil war and anarchy. Mostly being substanced by Citadel's chrasma. his son Aeos Citadel IV was bad 
% Vanyado conquerors and reincorruparates Nirvidium, now having been rebuilt into Nirvire

% 

\subsection{the invasions and reconquest}

\subsubsection{the prelude}
The year 8928 would mark a significant year on the continent of Vilterra. A united horde of dozens of Asakhani Tribes crossed
the mouth of the world into the formal territories of the Serovean Empire. Their leader Teljin 'the Wavemarker' sought to 
conquer the entire continent, which had still barely recovered from the Great Dragon War, 62 years before. However, for the 
kingdom of Rosinya it was only distance rumors. 

The aging Martim Vanyado was more focused on closer affairs. Their shakey allies, the Citadil's Remiant, had recently collapsed into civil war.
The Rosinyeans no longer trusting the remiant fully annexed Nirvidium, and sent templars north to prepare for another war with the Citadel. 
However, these anxiety melodied as by 8933 the Citadel's Remaniant completely collapsed into competing kingdoms. With the north
secured, it seemed like for a time Rosinya would have breathing room, but this peace would be short lived. 

In late 8937, the new Asakhan leader, Mahmud Ibn Teljin, secured a peaceful annexation of Junos and Vara. This put the
new empire right on Rosinya's border, sending a mixed reaction though the land. Many were ecsatic to witness the final
blow to high-elven rule, others were wary of these nomatic invaders. Martim was frantic in securing a diplomat mission to
the Asakhan. First he wanted to get in good terms with these new human rulers, second he hoped to convert Mahmud to 
the worship of Palid, and he finally wanted to secure Rosinyean lands on the Anjoi. 

It was late winter in early 8938, when the two leaders first met. They preformed extended talks in Junos for
much of the following year. Martim secured limited cooperation with Mahmud, insuring a temporary peace. 
Martim would allow free Asakhani passage through the Anjoi, while the Rosinyeans kept these lands. However,
Martim never secured any lasting relations, only aliening the Asakhan through his fervor. Either way,
way there was not much he could do. In 8942, Martim died of a stroke, leaving his regime in the hands of an 
unprepared ruling council. 

\subsubsection{the surge}
Everything quickly escalated in 8953. Mahmud Ibn Teljin is assassiated, leading to a two year power struggle 
between various elven and human factions in the Asakhan Empire. In 8955, Mahmud's 17 year old son Ahmet
barely secured the Asakhan throne. Within internal problem's surging through the empire, Ahmet wanted an
external victory to distract the country and secure his reign. Only a month into his reign, Ahmet launched
a surprise attack into Rosinya. 

The combined force of light Horse Archer, Wyverns, and Battlemages quickly overwelmed Rosinyean forces.
Within weeks, the Rosinyean's lost all territory along the Anjoi. Ahmet reoriented his forces to push
north, hoping to complete cripple Rosinya. Braceo surrended quickly, Ajetos and even the great Nirvire
fell within month, while most other intial Rosinyean forces were wiped out within half a year. 
Ahmet temporarily returned home to celebrate his victory. Remaining Rosinyean forces gathered in the 
in the last stronghold of Miridea. Preparing for a final stand in the war.  

% In 8922, the Asakhan under Telyin enter Vilterra. In 8930, Mahmud finishing conquest of the Seroveans. In 8945, Rosinya is conquered by the Asakhan
% the northern kingdom invigorated by Saint Carla of Rosinya led a decades long resistances. By 8997, Rosinya is mostly restored, although Agostino's
% anjoi conquest remain in Asakhan hands, Rosinye living their become the Vasq people. Joao Braceo I claiming descent from Martim Vanyado becoming 
% the first king of the current dynasty. Also palid vikings of the Suewel supported the defense of Rosinya. 

\subsection{the contempary era}


\end{document}
