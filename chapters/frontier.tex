\documentclass[../main.tex]{subfiles} 

\begin{document}
and the underground reservoirs.

However, the position and seasonal nature of
Kelibo makes it unsuitable for permanent residence, and the size of the
groundwater reservoir and the hostile surface made extracting water impractical.

%%%%%%%%%%%%%%%%%%%%%%%%%%%%%%%%%%%%%%%%%%%%%%%%%%%%%%%%%%%%%%%%%%%%%%%
%                               Ecology                               %
%%%%%%%%%%%%%%%%%%%%%%%%%%%%%%%%%%%%%%%%%%%%%%%%%%%%%%%%%%%%%%%%%%%%%%%
\section{Ecology}
Lacking the most important ingredient of life, the Frontier desert has
almost no life sparing a couple cacti and bushes on the edges. However,
there are some exceptions, especially on the Outer Frontier.

\subsection{Animals and Plants}
Heat-tolerant plants such as cacti and bushes exists in the more
hospitable Outer Frontier.

Camels also exists there, brought in by settlers. Most are domesticated,
though some do roam areas protected from monsters by people.

\subsection{Hostile Life}
Because intelligent life exert such little influence over the area, the
Frontier (especially the Inner Frontier) is filled with monsters.

The Frontier monsters are mostly reptilian and insectoid.
Most, such as basilisks, came down from the mountains. Nagas,
Ankhegs, and other beasts fight each other for sustenance.

\subsubsection{Elementals}
The most apt description of these mysterious, hostile creatures is
the manifestation of the forces of nature. Their origin are unknown,
their actions are inexplicable, all that's known about them is that
they will attack anything that they come across.

The elementals of the Frontier desert are mostly Fire, Air, and Earth.

\subsection{Sandworms}
Perhaps the most spectacular of the legends of the Frontier is the \emph{Sandworms},
and they certainly live up to it.

They are beings of massive proportions, some hundreds of feet long.
Much of their life is spent underground, where they supposedly
sustain themselves on underground water. The few times they surface
are primarilly for hunting, sunlight, and air.

Ironically, their times of hunting is also when they are most
vulnerable. Sandworms are prone to being prayed upon by other monsters.
The Shai-al run also use this opportunity to tame them.

Interestingly, Sandworms are attracted by music. The reason of which is
unknown. The Shai-al run takes advantage of this to call upon them.

\subsection{Migration of Kelibo}
At the north of the Frontier lies a seasonal lake Kelibo. In late winter,
early spring, the water from the northern mountains flows down into the
lake, attracting all kinds of life.

Basilisks from the mountains follows this stream down to the Frontier,
and some are left unable to leave.

Notably, the sandworms, despite having a consistent supply of underground
water, also gathers aroudn Kelibo primarily for hunting.

Because the allure of water is too great, bloody battles ensues everytime
water fills the lake. The violent nature of this seasonal event earned it
the name \emph{the Red Spring}.

Intelligent life such as the red dragonbornes also joins the fray for
sandworms; likewise, the Shai-al run are also present for water and
defense of the sandworms.

\section{History}
Not much ocurred in the Frontier worth noting; for the most part, it has
remained populated only by monsters and 



\end{document}
